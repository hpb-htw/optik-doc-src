\chapter{Allgemeine}

\asypmtote\cite{Asymptote} ist eine Programmiersprache um Graphiken, vor allen aber nicht ausschließlich geometrische Konstruktionen zu erstellen.
Das Paket \optik{} wurde mit dem Ziel, die Erstellung von gängigen optische Diagramme in der Programmiersprache \asypmtote{} zu erleichtern, geschrieben.


Beispielen in diesem Dokument sind in \repo{} zu finden. 
Ein Beispiel fängt --wenn nicht anders erwähnt-- mit folgenden Zeilen an:

\begin{verbatim}
settings.tex="lualatex";  // input Unicode character directly from keyboard
settings.outformat="pdf"; // use PDF as default output

texpreamble("\newcommand{\pointLable}{1}{\mathsf{#1}}
"); // optional

unitsize(2mm);                       // justify size / scale of image

import geometry;                     // Optik uses this package.
import "../optik.asy" as optik;      // the path to the file `optik.asy' must be adjusted.
\end{verbatim}


Der Inhalt der Datei \verb|shortcut.tex| ist wie folgt:

\begin{verbatim}
\newcommand{\pLabel}[1]{\mathsf{#1}}
\newcommand{\tLabel}[1]{\textsf{#1}}
\end{verbatim}


Die beiden obigen Code-Auszüge lassen darauf schließen, 
dass eine funktionierende Installation von \LaTeX{} und \asypmtote{} die notwendige Voraussetzung für die Nutzung des Paket \optik{} ist.


\includegraphics{asy/demo/demomirror}

\includegraphics{asy/demo/demoverticalmirror}


